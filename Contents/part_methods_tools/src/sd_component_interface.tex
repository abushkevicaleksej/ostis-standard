\begin{SCn}
\scntext{введение}{Проектирование интерфейса компьютерных систем — это один из наиболее важных этапов разработки любой системы. Пользователь при использовании интерфейса должен представить себе, какая информация о выполняемой задаче у него существует, и в каком состоянии находятся средства, с помощью которых он будет решать данную задачу. Эффективность работы пользователя и его интерес обеспечивает правильно сформулированная методика разработки и проектирования пользовательского интерфейса.
	В рамках главы рассмотрены этапы проектирования пользовательских интерфейсов и этапы проектирования
	адаптивных интеллектуальных мультимодальных пользовательских интерфейсов.}
\begin{scnrelfromlist}{дочерний раздел}
	\scnitem{§ 5.4.1. Анализ методик проектирования пользовательских интерфейсов}
	\scnitem{§ 5.4.2. Многократно используемые компоненты интерфейсов ostis-систем}
\end{scnrelfromlist}
\begin{scnrelfromlist}{ключевое понятие}
	\scnitem{библиотека многократно используемых компонентов пользовательских интерфейсов ostis-систем}
	\scnitem{метод оценки пользовательских интерфейсов}
	\scnitem{качественный метод оценки пользовательских интерфейсов}
	\scnitem{количественный метод оценки пользовательских интерфейсов}
	\scnitem{тестирование удобства использования пользовательских интерфейсов}
	\scnitem{отслеживание движения глаз}
	\scnitem{экспертная оценка пользовательских интерфейсов}
	\scnitem{А/Б тестирование пользовательских интерфейсов}
	\scnitem{древовидное тестирование пользовательских интерфейсов}
\end{scnrelfromlist}
\begin{scnrelfromlist}{библиографическая ссылка}
	\scnitem{Ehlert P.IntelUIIaS-2003bk}
	\scnitem{Kong J..Desig oHCAM-2011art}
	\scnitem{Sadouski M..Metho aTfCID-2022art}
	\scnitem{Ivory M..tState otAiAUEoUI-2001art}
	\scnitem{Jeffries R..UserIEitRW-1991art}
	\scnitem{ISOEoHSIPGoVUIE-2016el}
	\scnitem{Зенг В.А..ОценкКППИНП-2019ст}
	\scnitem{ISOTREoHSI-2002el}
	\scnitem{Корончик Д.Н.СеманТКПП-2011ст}
	\scnitem{Корончик Д.Н.ПользИИМП-2014ст}
\end{scnrelfromlist}
\scnheader{парадигма грамотного пользователя}
\begin{scnrelfromset}{базируется}
	\scnitem{знание управления системой}
	\scnitem{ответственность за качество взаимодействия с системой}
\end{scnrelfromset}
\scnheader{Технология OSTIS}
\scnidtf{комплексная технология проектирования интеллектуальных компьютерных систем нового поколения}
\scnidtf{технология, описывающая этапы проектирования пользовательских интерфейсов ostis-систем}
\begin{scnrelfromlist}{обеспечивает}
	\scnitem{высокая скорость разработки}
	\scnitem{создание удобных пользовательских интерфейсов}
	\scnitem{улучшение опыта использования интеллектуальной системы}
	\scnitem{повышение эффективности работы пользователей}
\end{scnrelfromlist}


\scnheader{основные этапы проектирования адаптивных интеллектуальных мультимодальных пользовательских интерфейсов в рамках работы Ehlert P.IntelUIIaS-2003bk}
\begin{scnrelfromlist}{разделение}
	\scnitem{анализ}
	\scnitem{разработка интерфейса}
	\scnitem{оценка интерфейса}
	\scnitem{доработка и усовершенствование}
\end{scnrelfromlist}
\scntext{пояснение}{Этап анализа является, вероятно, самой важной фазой в любом процессе проектирования, в том числе в проектировании интерфейсов ostis-систем. В процессе проектирования традиционного интерфейса необходимо проанализировать, кто является обычным пользователем, какие задачи интерфейс должен поддерживать.}


\scnheader{этап анализа}
\begin{scnrelfromlist}{разделение}
	\scnitem{функциональный анализ}
	\begin{scnindent}
		\scntext{значение}{в рамках данного этапа необходимо дать ответ на вопрос "каковы основные функции системы".}
	\end{scnindent}
	\scnitem{анализ данных}
	\begin{scnindent}
		\scntext{значение}{в рамках данного этапа необходимо определить значение и структуру данных, используемых в приложении.}
	\end{scnindent}
	\scnitem{анализ пользователей}
	\begin{scnindent}
		\scntext{значение}{данный этап обуславливает четкое выделение типов пользователей и их возможностей в интеллектуальном и когнитивном плане.}
	\end{scnindent}
	\scnitem{анализ среды}
	\begin{scnindent}
		\scntext{значение}{в рамках данного этапа необходимо определить требования, предъявляемые к среде, в которой будет работать система.}
	\end{scnindent}
\end{scnrelfromlist}
\begin{scnrelfromset}{результат}
	\scnitem{спецификация целей и информационных потребностей пользователя}
	\scnitem{спецификация функций и информации}
\end{scnrelfromset}


\scnheader{этап разработки интерфейса}
\begin{scnrelfromlist}{разделение}
	\scnitem{создание модели интерфейса в соответствии с этапом анализа}
	\scnitem{реализация модели интерфейса}
\end{scnrelfromlist}
\begin{scnrelfromset}{результат}
	\scnitem{пользовательский интерфейс}
\end{scnrelfromset}
\scntext{пояснение}{Результатом данного этапа является пользовательский интерфейс, который, по мнению разработчика, удовлетворяет требованиям пользователей и соответствует требованиям, сформулированным на этапе анализа.}

\scnheader{этап оценки интерфейса}
\begin{scnrelfromlist}{предполагает}
	\scnitem{удовлетворительные требования, сформированные на этапе анализа}
	\scnitem{исследованной эффективности модели интерфейса}
\end{scnrelfromlist}
\scntext{пояснение}{На этапе оценки интерфейса необходимо вернуться к требованиям этапа анализа. Требования, которые были сформулированы на этапе анализа, должны быть выполнены, а также должна быть исследована эффективность модели интерфейса. Чтобы определить эту эффективность, необходимо определить критерии эффективности.}

\scnheader{критерии эффективности}
\begin{scnrelfromlist}{разбиение}
	\scnitem{количество ошибок}
	\scnitem{время выполнения задачи}
	\scnitem{отношение пользователя к интерфейсу}
	\scnitem{и т.д.}
\end{scnrelfromlist}
\scntext{пояснение}{Доработка и усовершенствование осуществляется на основе проблем, выявленных на этапе оценки. В рамках данного этапа вносится ряд улучшений в модель интерфейса. Затем начинается новый цикл проектирования. Этот итеративный процесс будет продолжаться до тех пор, пока результат оценки не будет удовлетворять обозначенным требованиям.}


\scnheader{основные этапы проектирования адаптивных интеллектуальных мультимодальных пользовательских интерфейсов в рамках работы  Kong J..Desig oHCAM-2011art}
\begin{scnrelfromlist}{разделение}
	\scnitem{моделирование пользовательского интерфейса}
	\begin{scnindent}
		\scntext{значение}{описание абстрактного пользовательского интерфейса}
	\end{scnindent}
	\scnitem{проектирование пользовательского интерфейса по умолчанию}
	\begin{scnindent}
		\scntext{значение}{стандартная версия, конкретый пользовательский интерфейс}
	\end{scnindent}
	\scnitem{разработка пользовательского интерфейса}
	\begin{scnindent}
		\scntext{значение}{расширение или замена пользовательского интерфейса по умолчанию}
		\scntext{примечание}{этот шаг опускается, когда система генерирует пользовательский интерфейс по умолчанию автоматически}
	\end{scnindent}
	\scnitem{создание контекста использование}
	\begin{scnindent}
		\scntext{значение}{идентификация и создание контекста использования — модели пользователя, модель устройства и модель среды/платформы}
	\end{scnindent}
	\scnitem{адаптация пользовательского интерфейса}
	\begin{scnindent}
		\scntext{значение}{адаптация пользовательского интерфейса во время выполнения для соответствия конкретного контекста использования}
	\end{scnindent}
	\scnitem{кастомизация пользовательского интерфейса}
	\begin{scnindent}
		\scntext{значение}{ настройка пользовательского интерфейса самим пользователем (адаптируемость)}
	\end{scnindent}
\end{scnrelfromlist}

\scnheader{недостатки предложенных подходов}
\begin{scnrelfromlist}{разбиение}
	\scnitem{знания по каждому этапу проектирования находятся у разных специалистов в неформализорованном неунифицированном виде}
	\scnitem{отсутствие этапа формализованного документирования этапов проектирования приводит в дальнейшем к необходимости создания отдельных help-систем для пользователей, разработчиков и так далее}
\end{scnrelfromlist}

\scnheader{методика проектирование интерфейсов ostis-систем}
\begin{scnrelfromlist}{включает}
	\scnitem{анализ пользователя, его задач и целей, а также контекста использования}
	\begin{scnindent}
		\scntext{пояснение}{результаты первого этапа, такие как: модель конкретного пользователя, его потребности и контекст использования системы (устройство, окружающая среда) должны быть формализованы в рамках соответствующих онтологий базы знаний интерфейса. При этом в процессе формализации по необходимости должны быть переиспользованы компоненты базы знаний из библиотеку многократно используемых компонентов ostis-систем, а новые компоненты могут пополнить эту же библиотеку.}
	\end{scnindent}
	\scnitem{анализ требований к пользовательскому интерфейсу и спецификации проектируемого пользовательского интерфейса}
	\begin{scnindent}
		\scntext{пояснение}{Результатом второго этапа являются конечные требования к интерфейсу, которые должны быть сформулированы относительно модели пользователя и его цели, а также относительно контекста использования. Спецификация включает в себя список задач решаемых интерфейсом, описание внешних языков представления знаний. Результаты должны быть также формализованы, а в процессе выполнения могут быть использованы существующие компоненты из библиотеки многократно используемых компонентов интерфейсов ostis-систем.}
	\end{scnindent}
	\scnitem{задачно-ориентированная декомпозиция пользовательского интерфейса}
	\begin{scnindent}
		\scntext{пояснение}{На этапе задачно-ориентированной декомпозиции пользовательского интерфейса специфицированный интерфейс разбивается на интерфейсные подсистемы, которые могут разрабатываться параллельно. Это позволяет сократить сроки проектирования пользовательского интерфейса. Целесообразно проводить разбиение таким образом, чтобы максимальное количество подсистем уже имелось в библиотеке многократно используемых компонентов пользовательских интерфейсов ostis-систем.}
	\end{scnindent}
	\scnitem{проектирование пользовательского интерфейса по умолчанию}
	\begin{scnindent}
		\scntext{пояснение}{В соответствии с требованиями к пользовательскому интерфейсу, строится модель адаптивного интеллектуального мультимодального пользовательского интерфейса, которая является результатом третьего этапа. Такая модель будет включать в себя формализованную модель базы знаний и решателя задач. При проектировании могут быть использованы компоненты интерфейса, базы знаний и решателя задач. Компоненты будут записаны в унифицированном виде, что позволит обеспечить их автоматическую совместимость.}
	\end{scnindent}
	\scnitem{разработка пользовательского интерфейса}
	\begin{scnindent}
		\scntext{пояснение}{Результатом четвертого этапа является реализация спроектированного пользовательского интерфейса. После разработки пользовательского интерфейса выделяются типовые фрагменты интерфейса. Специфицируя фрагменты интерфейса необходимым образом следует включать их в библиотеку многократно используемых компонентов пользовательских интерфейсов ostis-систем. При разработке пользовательского интерфейса также можно использовать готовые компоненты интерфейса из библиотеки многократно используемых компонентов пользовательских интерфейсов ostis-систем.}
	\end{scnindent}
	\scnitem{анализ пользовательского интерфейса и его адаптация}
	\begin{scnindent}
		\scntext{пояснение}{На данном этапе используются готовые компоненты решателя задач: sc-агенты анализа пользовательского интерфейса и sc-агенты изменения модели пользовательского интерфейса на основе логических правил адаптации. Таким образом будет сформирована база знаний проектируемого интерфейса, которая автоматически может быть использована в качестве help-системы для пользователей, разработчиков и так далее.}
	\end{scnindent}
\end{scnrelfromlist}
\scntext{пояснение}{Поскольку знания о конкретном этапе обычно находятся у разных экспертов, особенностью предлагаемого подхода является обязательное формализованное документирование знаний в унифицированном виде и применение на каждом из этапов компонентного подхода.
	Для применения компонентного подхода предлагается использовать библиотеку многократно используемых компонентов ostis-систем.}


\scnheader{оценка пользовательского интерфейса}
\begin{scnrelfromlist}{необходима}
	\scnitem{улучшение коммуникации между интеллектуальными системами и их пользователями}
	\scnitem{время выполнения задачи}
	\scnitem{отношение пользователя к интерфейсу}
	\scnitem{и т.д.}
\end{scnrelfromlist}
\scntext{пояснение}{При оценке пользовательских интерфейсов большую роль играет человеческий фактор, а основными участниками являются пользователи и эксперты. Очень сложно оценить правильность решения по адаптации пользовательского интерфейса интеллектуальной системы и оценить его без участия человека.}


\scnheader{правила оценки пользовательских интерфейсов}
\begin{scnrelfromlist}{необходима}
	\scnitem{интерфейс должен быть интуитивно понятен для конечного пользователя}
	\scnitem{интерфейс должен быть доступным для пользователей с ограниченными возможностями и пользователей, впервые сталкивающимися с информационными технологиями}
	\scnitem{достижение цели пользователем должно осуществляться наименее возможным количеством шагов}
\end{scnrelfromlist}



\scnheader{методы оценки пользовательских интерфейсов интеллектуальных систем}
\begin{scnrelfromlist}{включают}
	\scnitem{оценка точности и полноты}
	\begin{scnindent}
		\scntext{пояснение}{Это метод, при котором оценивается точность и полнота ответов системы на запросы пользователей. 
			Точность означает, насколько хорошо пользовательский интерфейс отражает действительный опыт пользователя и предлагает ему наиболее подходящий контент в соответствии с его потребностями и предпочтениями. 
			Оценить точность возможно путем анализа, насколько легко использовать интерфейс, насколько он отражает действительные возможности интеллектуальной системы и насколько он помогает пользователям достигать своих целей более эффективно. 
			Полнота, с другой стороны, означает, насколько хорошо пользовательский интерфейс предоставляет всю необходимую информацию и функциональность, которые пользователь может потребовать в процессе взаимодействия с системой. Оценить полноту возможно путем анализа того, насколько полное и четкое описание функций, возможностей и ограничений системы предоставляется через пользовательский интерфейс, в каком объеме пользователю доступна необходимая информация для принятия решений и насколько система может эффективно реагировать на запросы и потребности пользователя}
	\end{scnindent}
	\scnitem{оценка персонализации}
	\begin{scnindent}
		\scntext{пояснение}{Это метод, при котором оценивается способность системы адаптироваться к потребностям и предпочтениям каждого пользователя. Оценка персонализации выполняется с помощью анализа пользовательских данных, а также проведения опросов среди пользователей, чтобы определить, насколько хорошо система учитывает их потребности.}
	\end{scnindent}
\end{scnrelfromlist}

\scnheader{измеряемые параметры оценки качества персонализации}
\begin{scnrelfromlist}{включает}
	\scnitem{работоспособность}
	\scnitem{уникальность}
	\scnitem{понятность}
	\scnitem{эффективность}
	\scnitem{удовлетворенность}
\end{scnrelfromlist}
\scntext{пояснение}{Оценка персонализации адаптивных пользовательских интерфейсов может быть выполнена различными методами, включая тестирование с использованием фокус-групп, опросы пользователей, анализ данных и другие. 
	Важно отметить, что оценка персонализации должна проходить на всех этапах разработки для повышения пользовательского опыта и улучшения качества пользовательских интерфейсов интеллектуальных систем.}

\scnheader{метод оценки пользовательских интерфейсов}
\begin{scnrelfromlist}{включает}
	\scnitem{качественный метод оценки пользовательских интерфейсов}
	\begin{scnrelfromlist}{разделение}
		\scnitem{тестирование удобства использования}
		\scntext{значение}{тестирование удобства использования пользовательских интерфейсов является важной частью процесса проектирования и разработки. тестирование удобства использования гарантирует, что эти интерфейсы удобны и полезны для потенциальных пользователей}
		\begin{scnrelfromlist}{этапы}
			\scnitem{определение целевых пользователей}
			\scnitem{разработка тестовых сценариев}
			\scnitem{проведение тестирования пользователей}
			\scnitem{анализ результатов тестирования}
			\scnitem{итерация дизайна}
		\end{scnrelfromlist}
		\scnitem{отслеживание движения глаз}
		\scntext{значение}{метод, который используется для анализа того, как пользователи взаимодействуют с пользовательским интерфейсом. Отслеживание движения глаз определяет точки фиксации взгляда пользователя при взаимодействии с системой, а также переходы между ними}
		\scnitem{экспертная оценка}
		\scntext{значение}{метод экспертной оценки пользовательских интерфейсов заключается в исследовании пользовательского интерфейса на соответствие заранее определенным правилам}
		\begin{scnrelfromlist}{этапы}
			\scnitem{выявление экспертов}
			\scnitem{проведение оценки}
			\scnitem{анализ результатов}
		\end{scnrelfromlist}
	\end{scnrelfromlist}
	\scnitem{количественный метод оценки пользовательских интерфейсов}
	\begin{scnrelfromlist}{разделение}
		\scnitem{А/Б тестирование}
		\scntext{значение}{ метод сравнения двух версий пользовательского интерфейса. Результатом проведения метода является выявление версии интерфейса наиболее подходящей для выполнения конкретной задачи}
		\begin{scnrelfromlist}{этапы}
			\scnitem{определение гипотезы}
			\scnitem{определение размеров выборов}
			\scnitem{итерация интерфейса}
			\scnitem{наблюдение и сбор данных}
			\scnitem{анализ результатов}
		\end{scnrelfromlist}
	\end{scnrelfromlist}
	\scnitem{древовидное тестирование}
	\scntext{значение}{метод оценки качества пользовательских интерфейсов, который заключается в тестировании древовидной структуры навигации по системе}
	\begin{scnrelfromlist}{этапы}
		\scnitem{создание виртуального интерфейса}
		\scnitem{выполнение тестового задания}
		\scnitem{получение прав доступа}
		\scnitem{фиксирование действий}
		\scnitem{анализ путей}
		\scnitem{формирование рекомендации по оптимизации}
	\end{scnrelfromlist}
\end{scnrelfromlist}



\scnheader{проблема хранения и поиска многократно используемых компонентов интерфейсов ostis-систем}
\begin{scnrelfromlist}{формируется}
	\scnitem{большое число многократно используемых компонентов интерфейсов ostis-систем}
\end{scnrelfromlist}

\scnheader{решение проблемы хранения и поиска многократно используемых компонентов интерфейсов ostis-систем}
\begin{scnrelfromlist}{включает}
	\scnitem{библиотека многократно используемых компонентов пользовательских интерфейсов ostis-систем}
	\scnitem{менеджер многократно используемых компонентов ostis-систем}
\end{scnrelfromlist}

\scnheader{классификация компонентов библиотеки}
\begin{scnrelfromlist}{разбиение}
	\scnitem{просмотрщик содержимого файлов ostis-системы}
	\scnitem{редактор содержимого файлов ostis-системы}
	\scnitem{транслятор содержимого файла ostis-системы в SC-код}
	\scnitem{транслятор из SC-кода в содержимое файла ostis-системы}
	\scnitem{транслятор базы знаний во внешнее представление}
\end{scnrelfromlist}
\scntext{пояснение}{Технология OSTIS позволяет интегрировать в качестве компонентов редакторы и просмотрщики, разработанные с использованием других технологий (далее их будем называть платформенно-зависимыми многократно используемыми компонентами интерфейса ostis-систем). В основном они используются для просмотра и редактирования содержимого файлов ostis-системы. Это значительно позволяет сэкономить время при их разработке}


\scnheader{использование библиотеки многократно используемых компонентов интерфейсов ostis-систем при проектировании интерфейса прикладной системы}
\begin{scnrelfromlist}{обеспечивает}
	\scnitem{сокращение сроков проектирования}
	\scnitem{снижение требований к разработчику}
\end{scnrelfromlist}
\scntext{пояснение}{Это достигается за счет проектирования
	интерфейса из уже заранее подготовленных моделей интерфейса, что также позволяет повысить качество проектируемого интерфейса}

\scnheader{этапы проектирования, рассмотренные в данной главе}
\begin{scnrelfromlist}{разбиение}
	\scnitem{анализ пользователя, его задач и целей, а также контекста использования}
	\scnitem{анализ требований к пользовательскому интерфейсу и спецификация проектируемого пользовательского интерфейс}
	\scnitem{задачно-ориентированная декомпозиция пользовательского интерфейса}
	\scnitem{проектирование пользовательского интерфейса по умолчанию}
	\scnitem{разработка пользовательского интерфейса}
	\scnitem{анализ пользовательского интерфейса и его адаптация}
\end{scnrelfromlist}
\scntext{заключение}{Одной из ключевых особенностей предлагаемой методики является использование многократно используемых
	компонентов интерфейсов, что позволяет существенно сократить время проектирования, а также уменьшить требования к профессиональной квалификации разработчика}


\end{SCn}